\documentclass{article}
%==============================================================================%
%	                          Packages                                     %
%==============================================================================%
% Packages
\usepackage[utf8]{inputenc}
\usepackage{graphicx}
\usepackage{float}
\usepackage{amsmath}
\usepackage{amssymb}
\usepackage{braket}
\usepackage{subcaption}
\usepackage[margin=0.7in]{geometry}
\usepackage[version=4]{mhchem}
%==============================================================================%
%                           User-Defined Commands                              %
%==============================================================================%
% User-Defined Commands
\newcommand{\be}{\begin{equation}}
\newcommand{\ee}{\end{equation}}
\newcommand{\benum}{\begin{enumerate}}
\newcommand{\eenum}{\end{enumerate}}
\newcommand{\pd}{\partial}
\newcommand{\dg}{\dagger}
\newcommand{\sumzero}{\sum_{n=0}^\infty}
\newcommand{\sumone}{\sum_{n=1}^\infty}
\newcommand{\bA}{\mbox{\bf A}}
%==============================================================================%
%                             Title Information                                %
%==============================================================================%
\title{Chem237: Lecture 13}
\date{5/3/18}
\author{Shane Flynn}
%==============================================================================%
%	Everyone Please Make Comments if Something Needs to be Reviewed        %
%                           Or just fix it yourself!                           %
%==============================================================================%
\begin{document}
\maketitle

\section{Note}
9-9-19; Just copied the lecture notes I have over, this has not been reviewed/edited/made ready for the world. 

\section*{Gaussian Elimination}
Reduce your matrix to an upper-triangular (or lower triangular) form by performing operations such that the determinant doesn't change. 
%==============================================================================%
%                               Note/Question:
%(Are we using bold capital letters for matricies? let's be consistent with notation). 
%==============================================================================%

Consider the square N by N matrix $\bA$, which has column vectors labeled $\vec{a}_i$

\be
\bA =  
\begin{bmatrix}
    A_{11}  & \dotsb &  A_{1N} \\
    \vdots  & \ddots &  \vdots \\
    A_{1N}  & \dotsb &  A_{NN} 
\end{bmatrix}
    = 
\begin{bmatrix}
        \vec{a}_1 & \dots  & \vec{a}_N 
\end{bmatrix}
\ee

To compute the determinant of $\bA$ we know the following
\be
\det\left(\bA\right) = \det\left(\vec{a}_1 \cdots  \vec{a}_N \right) = \det\left(\vec{a}_1- \lambda_{a2}, \vec{a}_2 \cdots \vec{a}_N \right) 
\ee

Let $\lambda_1 = \frac{A_{N1}}{A_{N2}}$ then we have
\be
\begin{bmatrix}
    \star  & A_{12} & \dotsb &  A_{1N} \\
    \star & A_{22} &\dotsb &  \vdots \\
    \vdots & \vdots &  \ddots &  \vdots \\
    0  & A_{N2} & \dotsb &  A_{NN} 
\end{bmatrix}
\ee

Next step set $\lambda_2 = \frac{A_{N2}}{A_{N3}}$ then we have
\be
\begin{bmatrix}
    \star  & \star & \dotsb & \vdots \\
    \star & \vdots &\dotsb &  \vdots \\
    \vdots & \vdots &  \ddots &  \vdots \\
    0  & 0 & \dotsb &  A_{NN} 
\end{bmatrix}
\ee

In this way we can reduce every element of the Nth row to 0 except for the final term A$_{NN}$.
We can then repeat this process for the N-1 row (Second to last row), which will convert every element up to N-1 in the row to 0.%you want to keep last 0 in Nth row so only go up to N-1 in N-1 row
\be
\begin{bmatrix}
    \star  & \star & \dotsb & \vdots \\
    \star & \vdots &\dotsb &  \vdots \\
    0 & \dotsb &  0 &  \star \\
    0  & \dotsb & 0 &  A_{NN} 
\end{bmatrix}
\ee

Repeating the process iteratively eventually ends in
\be
\begin{bmatrix}
    \star  & \star & \star & \star \\
     0 & \star &\star &  \star \\
     0 & 0 &  \star &  \star \\
    0  & 0 & 0 &  \star 
\end{bmatrix}
\ee

What happens if you have a 0 at some point during the iterations? Just move it and keep going. 

\subsection{Numerical Linear Algebra}
Using the definition of the determinant the scaling would be N! which is very bad. 
Using the algorithm from above leads to N$^3$ scaling, which is a common result for numerical linear algebra algorithms.
$a_1$ - $\lambda_{a2}$ = 2 operations scaling as N, Appy to row = N, apply to other ros = N. 

\subsection{Linear Indepedence}
Consider a set of n vectors $\vec{a}_1 \cdots \vec{a}_N$. 
These vectors are linearly indepedent if for any $\lambda_1\cdots\lambda_N$
\be
\sum_{i=1}^N \lambda_i\vec{a}_i \neq 0
\ee
and vice versa linearly depedent if 
\be
\sum_{i=1}^N \lambda_i\vec{a}_i = 0
\ee

Theorem: determinate matrix formed by N vectors det($\vec{a}_1\cdots\vec{1}_N$) = 0 iff $\vec{a}_1\cdots\vec{a}_N$ are linearly depedent, and vice versa linearly indepedent vectors produce a determinate $\neq$ 0. 




\end{document}
