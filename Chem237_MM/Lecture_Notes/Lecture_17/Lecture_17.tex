\documentclass{article}
%==============================================================================%
%	                          Packages                                     %
%==============================================================================%
% Packages
\usepackage[utf8]{inputenc}
\usepackage{graphicx}
\usepackage{float}
\usepackage{amsmath}
\usepackage{amssymb}
\usepackage{braket}
\usepackage{mathtools}
\usepackage{subcaption}
\usepackage[margin=0.7in]{geometry}
\usepackage[version=4]{mhchem}
%==============================================================================%
%                           User-Defined Commands                              %
%==============================================================================%
% User-Defined Commands
\newcommand{\be}{\begin{equation}}
\newcommand{\ee}{\end{equation}}
\newcommand{\benum}{\begin{enumerate}}
\newcommand{\eenum}{\end{enumerate}}
\newcommand{\pd}{\partial}
\newcommand{\dg}{\dagger}
\newcommand{\sumzero}{\sum_{n=0}^\infty}
\newcommand{\sumone}{\sum_{n=1}^\infty}
%==============================================================================%
%                             Title Information                                %
%==============================================================================%
\title{Chem237: Lecture 17}
\date{5/15/18}
\author{Alan Robledo}
%==============================================================================%
%	Everyone Please Make Comments if Something Needs to be Reviewed        %
%                           Or just fix it yourself!                           %
%==============================================================================%
\begin{document}
\maketitle
%==============================================================================%
% The end of integral transforms lecture should go here.
%==============================================================================%
\section*{Solving Linear ODEs using Matrix Methods}
\subsection{First Order ODE}
We've already discussed solving ODEs (Ordinary Differential Equations) in previous lectures as well as performing an eigenvalue decomposition.
Therefore, it makes sense to discuss how we can combine the two topics.
We will start with a quick example of a first-order ODE,
\be
  \dot{y} = a y \quad ; \quad y = y(t)
\ee
where we are adopting the dot notation from physics where a dot represents a derivative taken with respect to time, e.g. $\dot{y} = \frac{dy}{dt}$, for simplicity.

By inspection, the solution space is one dimension and has the form
\be
  y = y_o e^{a t}
\ee
where a is a scalar and $y_o$ would be determined from intitial conditions.

We can consider generalizing this by considering an array of functions
\be
  \vec{Y}(t) =
  \begin{bmatrix}
    y_1(t) \\
    \vdots \\
    y_n(t) \\
  \end{bmatrix}
\ee
and a matrix of coefficients \textbf{A}
\be
  \begin{bmatrix}
    A_{11} & \cdots & A_{n1} \\
    \vdots & & \vdots \\
    A_{1n} & \cdots & A_{nn}\\
  \end{bmatrix}
\ee
such that the following equation is satisfied
\be \label{eq:gen_firstode}
  \dot{Y}(t) = \textbf{A} Y(t) .
\ee
It is implied that $Y = \vec{Y}$.
Equation \ref{eq:gen_firstode} looks very similar to the Time-Dependent Schr\"odinger Equation (TDSE),
\be
  i \hbar \frac{\partial \Psi}{\partial t} = \hat{H} \Psi
\ee
and, without using any quantum mechanics jargon, we can compare this to equation 1 and say that the solution to the TDSE is,
\be
  \Psi = \psi_o e^{-\frac{it}{\hbar} \hat{H}}
\ee
If this does bother you then we'll just assume that $\psi_o$ is a stationary state.
Going back to equation \ref{eq:gen_firstode}, we can automatically see that the solution looks like,
\be \label{eq:gen_solution}
  Y(t) = e^{\textbf{A} t} Y_o
\ee
where $Y_o = \begin{bmatrix} y_1(0) \\ \vdots \\ y_n(0) \end{bmatrix}$.
If we perform an eigenvalue decomposition on \textbf{A}, then we get,
\be \label{eq:eigen_decomp}
  \textbf{A} = \textbf{X} \Lambda \textbf{X}^{-1}
\ee
And using this with equation \ref{eq:gen_solution} gives
\be \label{eq:eigen_solution}
  Y(t) = \textbf{X} e^{\Lambda t} \textbf{X}^{-1} Y_o
\ee
where we recall that \textbf{X} is a matrix of eigenvectors and $\Lambda$ is a diagonal matrix of the eigenvalues.
Remember that $Y_o$ is a vector and not a scalar, so it is important to keep it on the right hand side of the equation so that the dimensions are aligned properly.

If you think that there was a big jump between equation \ref{eq:eigen_decomp} and equation \ref{eq:eigen_solution}, we are simply taking the idea from lecture 14 that \textbf{A} and $e^{\textbf{A}}$ share the same eigenvectors while the eigenvalues of \textbf{a} and $e^{\textbf{A}}$ are $\lambda_n$ and $e^{\lambda_n}$ respectively (see lecture 14 for the proof).
Likewise, if we mutliply \textbf{A} by a scalar, t, then \textbf{A}t and $e^{\textbf{A} t}$ will share the same eigenvectors while the eigenvalues of \textbf{A}t and $e^{\textbf{A} t}$ are $\lambda_n$t and $e^{\lambda_n t}$ respectively, since scalar multiplication does not change the eigenvectors only the eigenvalues.
Therefore, using right multiplication,
\be
  e^{\textbf{A} t} \textbf{X} = \textbf{X} e^{\Lambda t} \quad \Rightarrow \quad e^{\textbf{A} t} = \textbf{X} e^{\Lambda t} \textbf{X}^{-1}
\ee
\subsection{General ODE}
Now, what if we had a system of linear ODEs of arbitrary order of the form
\be
  b_k y^{(k)} + b_{k-1} y^{(k-1)} + \dots + b_o y^{(0)} = 0
\ee
Recall that $y^{(k)}$ is the kth derivative of y.
We can introduce vectors
\begin{align}
    y_1 &= y^{(0)} \\
    y_2 &= y^{(1)} \\
    & \vdotswithin{ = }\notag \\
    y_k &= y^{(k-1)} \\
\end{align}
in this way so we can say,
\begin{align}
    \dot{y}_1 &= y_2 \\
    \dot{y}_2 &= y_3 \\
    & \vdotswithin{ = }\notag \\
    \dot{y}_{k-1} &= y_k \\
    \dot{y}_k &= \frac{-1}{b_k} (b_{k-1} y_k + b_{k-2} y_{k-1} + \dots + b_o y_1) \\
\end{align}
The coefficient matrix \textbf{A} is,
\be
  \begin{bmatrix}
    0 & 1 & 0 & \cdots & 0 \\
    0 & 0 & 1 & \cdots & 0 \\
     & & \vdots & & \\
    0 & 0 & 0 & \cdots & 1 \\
    \frac{-b_o}{b_k} & \frac{-b_1}{b_k} & \frac{-b_2}{b_k} & \cdots & \frac{-b_{k-1}}{b_k} \\
  \end{bmatrix}
\ee
\section{Partial Differential Equations}
Start with the simplest PDE.
Consider a function $U = U(x,t)$.
\be
  \frac{\partial U}{\partial t} + a \frac{\partial U}{\partial x} = 0
\ee
The solution to the equation is,
\be
  U(x,t) = f(x - at)
\ee
Next exmaple, a is no longer a constant but a function of x, i.e. $a = a(x)$.
Forward translation equation:
\be
  \frac{\partial U}{\partial t} + \frac{\partial (a(x) U)}{\partial x} = 0
\ee
Backward translation equation:
\be
    \frac{\partial U}{\partial t} - a(x) \frac{\partial U}{\partial x} = 0
\ee
Method to solve the forward translation equation is to use a characteristic curve.
Consider
\be \label{eq:define_a}
  \frac{dx}{dt} = a(x)
\ee
Now plot all solutions $x(t)$ that solve the equation above and plot them.
%==============================================================================
% Add plot of curves here
%==============================================================================
These are your trajectories all over the coordinate space.
Now we will define $U = U(x(t),t)$ so that $U$ is explicitly a function of t only.
Then we will take the total derivative of U,
\be
  \begin{split}
    \frac{d U}{d t} &= \frac{\partial U}{\partial t} \frac{\partial t}{\partial t} + \frac{\partial U}{\partial x} \frac{\partial x(t)}{\partial t} \\
    &= \frac{\partial U}{\partial t} + \frac{\partial U}{\partial x} a(x(t))
  \end{split}
\ee
where we plugged in equation (\ref{eq:define_a}) for the derivative of x with respect to t.
We can also plug in the forward translation equation for $\frac{\partial U}{\partial t}$.
\be
  \begin{split}
    \frac{d U}{d t} &= \frac{\partial U}{\partial t} + \frac{\partial U}{\partial x} a(x(t)) \\
    &= - \frac{\partial (a(x) U)}{\partial x} + a(x(t)) \frac{\partial U}{\partial x} \\
    &= - \frac{\partial a}{\partial x} U - \frac{\partial U}{\partial x} a + \frac{\partial U}{\partial x} a \\
    &= - \frac{\partial a(x(t))}{\partial x} U \\
    &= - a'(x(t)) U
  \end{split}
\ee
We can solve for $U$ by using separation of variables and integrating both sides.
\be
  \int \frac{d U}{U} = - \int a'(x(t)) dt
\ee
\be \label{eq:pde_soln}
  U(t) = U(0) e^{- \int_0^t a'(x(t)) dt}
\ee
\subsection{Example}
Consider $a(x) = x$.
We can define our characteristic curve as
\be
  \frac{\partial x}{\partial t} = x \quad \Longrightarrow \quad x(t) = x_o e^t
\ee
Then we can obtain our solution using equation (\ref{eq:pde_soln})
\be
  U(x,t) =
\ee
\end{document}
