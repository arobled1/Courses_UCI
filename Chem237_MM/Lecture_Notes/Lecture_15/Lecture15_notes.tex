\documentclass{article}
%=============================================================================80
%	                          Packages                                     %
%==============================================================================%
% Packages
\usepackage[utf8]{inputenc}
\usepackage{graphicx}
\usepackage{amsmath}
\usepackage{amssymb}
\usepackage{braket}
\usepackage{float}
\usepackage{subcaption}
\usepackage[margin=0.7in]{geometry}
\usepackage[version=4]{mhchem}
%==============================================================================%
%                           User-Defined Commands                              %
%==============================================================================%
% User-Defined Commands
\newcommand{\be}{\begin{equation}}
\newcommand{\ee}{\end{equation}}
\newcommand{\benum}{\begin{enumerate}}
\newcommand{\eenum}{\end{enumerate}}
\newcommand{\pd}{\partial}
\newcommand{\dg}{\dagger}
%==============================================================================%
%      Just for Lecture 15                                                     %
%==============================================================================%
\newcommand{\br}{\textbf{r}}
\newcommand{\bp}{\textbf{p}}
\newcommand{\bm}{\textbf{M}}
\newcommand{\bk}{\textbf{K}}
%==============================================================================%
%                             Title Information                                %
%==============================================================================%
\title{Chem237: Lecture 15}
\date{5/9/18}
\author{Alan Robledo}
%==============================================================================%
\begin{document}

\maketitle

\section{Classical Normal Mode Analysis}
%==============================================================================
%         Shane, you can probably make a better introduction
%==============================================================================
If we had a problem of a mass attached to a spring, we can immediately find the equations of motion from the harmonic oscillator model.
However, we can always imagine a situation where our system consists of having multiple masses connected to each other by springs.
For example, a water molecule \ce{H_2O} can be thought of as 3 atoms connected to each other by 2 springs.
In this system of coupled osciallators, the equations of motion that correspond to the molecular vibrations can be solved as if it were a coupled harmonic oscillator problem.
Solving this problem using cartesian coordinates can be incredibly tedious and physically impossible if we keep adding more atoms to the system.

This can be made much simpler by performing a transformation from the cartesian basis to the normal mode basis.
In the normal mode basis, we can imagine all of our atoms oscillating sinusoidally each with a frequency and phase relation that corresponds to a particular normal mode.
Chemists use normal mode analysis to determine whether a normal mode is IR and/or Raman active, which is useful when creating vibrational spectra.

In this lecture, we will show the mathematical basis for normal mode analysis.
But before we introduce how it can be applied to a quantum system such as atoms in a molecule, we will start with the classical analog.

\subsection{Harmonic Approximation}
We begin by writing the Hamiltonian for our system of interest with N vibrational degrees of freedom,
\be
  H = \sum_{i = 1}^{N} \frac{p^2_i}{2 m} + V(r_1, \dots, r_N)
\ee
where our coorindate vector is $\br = \begin{bmatrix} r_1 \\ \vdots \\ r_N \end{bmatrix}$.
We can invoke the \textbf{Harmonic approximation}, which says that the potential $V(\br)$ can be expressed as a taylor expansion about a minimum $r_o$, which we will take be 0,
\be
  V(r) = V(r_o) + \sum_j \Big( \frac{\partial V}{\partial r_j} \Big) r_j + \frac{1}{2} \sum_{i j} r_i K_{i j} r_j + \dots
\ee
where $\bk$ is called the \textbf{Hessian} and is defined as
\be
  K_{i j} = \frac{\partial^2 V}{\partial r_i r_j} .
\ee
Since our potential is expanded about a minimum, we can say that $\left. \frac{\partial V}{\partial r_j} \right|_{r_j = r_o} = 0$.
If we truncate at the quadratic term and assume that $V(r_o) \approx 0$, then we obtain an approximate potential
\be
  V(r) \approx \frac{1}{2} \sum_{i j} r_i K_{i j} r_j .
\ee
The Hamiltonian can then be written in terms of matrices as
\be \label{eq:gen_hamil}
  \textbf{H} \approx \frac{1}{2} \bp^T \bm^{-1} \bp + \frac{1}{2} \textbf{r}^{T} \bk \textbf{r}
\ee
By doing this, we have effectively turned our system into one where each degree of freedom can be approximated as a harmonic oscillator.
The equations of motion for each degree of freedom can then be solved by using Hamilton's equations:
\be
  \begin{split}
    \dot{r}_i &= \frac{\partial H}{\partial p_i} \\
    \dot{p}_i &= - \frac{\partial H}{\partial r_i} \\
  \end{split}
\ee
However, this can be difficult when we have a system of coupled oscillators.

\subsection{Normal Mode Coordinates}
One way to get around this problem is by transforming our original coordinates into a set of normal mode coordinates.
This decouples all of our degrees of freedom and turns our system into one of N independent harmonic oscillators, each with a normal mode frequency $\omega_i$.
If our mass matrix \textbf{M} is not diagonal, we could perform an eigenvalue decomposition and use the corresponding eigenvectors to find a diagonal matrix for \bm, which would just be a matrix of its eigenvalues.
Instead, we will introduce mass scaled coordinates to get rid of our mass matrix from the Hamiltonian.
These coordinates can be defined as:
\be
  \begin{split}
    r'_i = \sqrt{m_i} r_i \quad &\Longleftrightarrow \quad \br' = \bm^{1/2} \br \\
    p'_i = \frac{1}{\sqrt{m_i}} p_i \quad &\Longleftrightarrow \quad \bp' = \bm^{-1/2} \bp \\
    K'_{i j} = \frac{1}{\sqrt{m_i m_j}} K_{i j} \quad &\Longleftrightarrow \quad \bk' = \bm^{-1/2} \bk \bm^{1/2}
  \end{split}
\ee
So \textbf{H} becomes,
\be
  \textbf{H} = \frac{1}{2} \bp'^T \bp' + \frac{1}{2} \br'^T \bk' \br'
\ee
where \bk' is now our mass-scaled Hessian.

We can simplify the Hamiltonian even further by turning the mass-scaled Hessian into its diagonal form using an eigendecomposition,
\be
  \bk' = \textbf{U} \Lambda \textbf{U}^{\dagger} = \sum_{i = 1}^N \lambda_i U_i U_i^{\dagger}
\ee
where $\lambda_i$ is the ith eigenvalue of the Hessian and \textbf{U} is a matrix of the eigenvectors.
We will also make use of the \textbf{resolution of the identity},
\be
  \textbf{U} \textbf{U}^{\dagger} = \textbf{I} \Longleftrightarrow \textbf{I} = \sum_i U_i U_i^{\dagger}
\ee
which is a neat trick that is used very often in quantum mechanics.
We can insert the identity matrix anywhere in our equation because it is like multiplying by 1.
\be
  \textbf{H} = \frac{1}{2} \bp'^T \bp' + \frac{1}{2} \br'^T \bk' \br' \quad \longrightarrow \quad \textbf{H} = \frac{1}{2} \bp'^T \textbf{U} \textbf{U}^{\dagger} \bp' + \frac{1}{2} \br'^T \textbf{U} \Lambda \textbf{U}^{\dagger} \br'
\ee
Then we will define the normal mode coordinates and momenta as
\be
  \tilde{\br} = \textbf{U}^{\dagger} \br' \quad ; \quad \tilde{\bp} = \textbf{U}^{\dagger} \bp'
\ee
which gives us the simplified Hamiltonian,
\be
  \begin{split}
    \textbf{H} &= \frac{1}{2} \tilde{\bp}^T \tilde{\bp} + \frac{1}{2} \tilde{\br}^T \Lambda \tilde{\br} \\
    &= \sum_{i=1}^{N} \Big( \frac{1}{2} \tilde{p}_i^2 + \frac{\lambda_i}{2} \tilde{r}_i^2 \Big)
  \end{split}
\ee
where each normal mode oscillates with a single frequency $\omega_i = \sqrt{\lambda_i}$.
\subsection{Example}
Consider an example of two classical masses attached to three springs as shown in the figure below.
%==============================================================================
 % Somebody should make a figure that looks like the drawing Vlad made in class
 % It's the same as the one in this website:
 % http://scienceworld.wolfram.com/physics/SpringsThreeSpringsandTwoMasses.html
%==============================================================================
The Hamiltonian for the system is simple to write because we just have coupled harmonic oscillators.
\be
  H = \frac{p_1^2}{2 m_1} + \frac{p_2^2}{2 m_2} + \frac{k_1}{2}r_1^2 + \frac{k_2}{2}(r_1 - r_2)^2 + \frac{k_3}{2}r_2^2
\ee
For simplicity, let's just say that the spring constants are the same, i.e. $k_1 = k_2 = k_3 = k$, and the masses are the same, i.e. $m_1 = m_2 = m$.
So now our Hamiltonian simplifies to,
\be
  \begin{split}
    H &= \frac{p_1^2}{2 m} + \frac{p_2^2}{2 m} + \frac{k}{2}r_1^2 + \frac{k}{2}(r_1 - r_2)^2 + \frac{k}{2}r_2^2 \\
    &= \frac{1}{2m} \Big( p_1^2 + p_2^2 \Big) + \frac{k}{2} \Big( 2r_1^2 - 2r_1 r_2 + 2r_2^2 \Big)
  \end{split}
\ee
where $r_1$ and $r_2$ are each mass's coordinates measured from the equilibrium position.
By creating mass, coordinate, momenta, and Hessian matrices, we can make this look like equation (\ref{eq:gen_hamil}),
\be
\begin{split}
  \textbf{H} &= \frac{1}{2} \bp^T \bm^{-1} \bp + \frac{1}{2} \textbf{r}^{T} \bk \textbf{r} \\
  &= \frac{1}{2}
  \begin{bmatrix}
    p_1 \\
    p_2 \\
  \end{bmatrix}^T
  \begin{bmatrix}
    m & 0 \\
    0 & m \\
  \end{bmatrix}^{-1}
  \begin{bmatrix}
    p_1 \\
    p_2 \\
  \end{bmatrix}
   + \frac{1}{2}
   \begin{bmatrix}
     r_1 \\
     r_2 \\
   \end{bmatrix}^T
   \begin{bmatrix}
     2k & -k \\
     -k & 2k \\
   \end{bmatrix}
   \begin{bmatrix}
     r_1 \\
     r_2 \\
   \end{bmatrix}
 \end{split}
\ee
The mass matrix is simply a diagonal matrix because there are no cross terms in the kinetic energy.
However, the cross terms in the Hessian come from the $(r_1 - r_2)^2$ term in the potential.
If you're not sure where each matrix element came from, recall equation (8),
\be
  K_{i j} = \frac{\partial^2 V}{\partial r_i r_j} .
\ee
Since the potential is equal to,
\be
  V = \frac{k}{2} \Big( 2r_1^2 - 2r_1 r_2 + 2r_2^2 \Big)
\ee
we can say that the matrix elements for the Hessian are,
\be
  K_{11} = \frac{\partial^2 V}{\partial r_1^2} = 2k \quad ; \quad K_{12} =  \frac{\partial^2 V}{\partial r_1 r_2} = K_{21} = -k \quad ; \quad K_{22} = \frac{\partial^2 V}{\partial r_2^2} = 2k .
\ee
You can verify that equation 15 and 16 are the same.

Now we can define the mass-scaled Hessian,
\be
  \bk' =
  \begin{bmatrix}
    \frac{2 k}{\sqrt{m_1^2}} & - \frac{k}{\sqrt{m_1 m_2}} \\
    - \frac{k}{\sqrt{m_1 m_2}} & \frac{2k}{\sqrt{m_2^2}} \\
  \end{bmatrix}
  =
  \begin{bmatrix}
    \frac{2 k}{m} & - \frac{k}{m} \\
    - \frac{k}{m} & \frac{2k}{m} \\
  \end{bmatrix}
\ee
and find it's eigenvalues and eigenvectors. This can be easily done with Mathematica, but we can also do this by hand since our system is 2 dimensional. To find the eigenvalues, we set up the characteristic equation,
\be
  \text{det}(\bk' - \lambda \textbf{I}) = 0 \Longrightarrow \Big(\frac{2k}{m} - \lambda \Big)^2 - \Big( \frac{k}{m} \Big)^2 = 0 \Longrightarrow \Big( \lambda - \frac{2k}{m} \Big)^2 - \Big( \frac{k}{m} \Big)^2 = 0
\ee
and solve for $\lambda$.
\be
  \begin{split}
  \lambda - \frac{2k}{m} =& \pm \frac{k}{m} \Longrightarrow \lambda = \frac{2k}{m} \pm \frac{k}{m} \\
  &\lambda_1 = \frac{3k}{m} , \lambda_2 = \frac{k}{m}
  \end{split}
\ee
Now we find the eigenvector for each eigenvalue.

For $\lambda_1$, we have
\be
  \bk' \textbf{x} = \lambda_1 \textbf{x} \Longrightarrow (\bk' - \lambda_1 \textbf{I}) \textbf{x} = \textbf{0} \Longrightarrow
  \begin{bmatrix}
    \frac{2k}{m} - \lambda_1 & \frac{-k}{m} \\
    \frac{-k}{m} & \frac{2k}{m} - \lambda_1 \\
  \end{bmatrix}
  \begin{bmatrix}
    x_1 \\
    x_2 \\
  \end{bmatrix}
  =
  \begin{bmatrix}
    0 \\
    0 \\
  \end{bmatrix}
\ee
which is just solving two equations with two unknowns.
\be
  \begin{split}
    \frac{-k}{m}x_1 - \frac{k}{m}x_2 &= 0 \\
    \frac{-k}{m}x_1 - \frac{k}{m}x_2 &= 0 \\
  \end{split}
\ee
Since both equations are the same, we are really only left with one equation and two unknowns. To get around the problem, we can assign any value to $x_1$, like $x_1 = 1$, and just solve for $x_2$.
\be
  \frac{-k}{m}(1) - \frac{k}{m}x_2 = 0 \Longrightarrow x_2 = -1
\ee
So our eigenvector for $\lambda_1$ is $\begin{bmatrix} 1 \\ -1 \\ \end{bmatrix}$. If you don't understand why I was able to do this, refer to Lecture 14.

For $\lambda_2$, we have
\be
  \bk' \textbf{x} = \lambda_2 \textbf{x} \Longrightarrow (\bk' - \lambda_2 \textbf{I}) \textbf{x} = \textbf{0} \Longrightarrow
  \begin{bmatrix}
    \frac{2k}{m} - \lambda_2 & \frac{-k}{m} \\
    \frac{-k}{m} & \frac{2k}{m} - \lambda_2 \\
  \end{bmatrix}
  \begin{bmatrix}
    x_1 \\
    x_2 \\
  \end{bmatrix}
  =
  \begin{bmatrix}
    0 \\
    0 \\
  \end{bmatrix} .
\ee
So our two equations are:
\be
  \begin{split}
    \frac{k}{m}x_1 - \frac{k}{m}x_2 &= 0 \\
    \frac{-k}{m}x_1 + \frac{k}{m}x_2 &= 0 \\
  \end{split}
\ee
The bottom equation is just the top equation multiplied by $-1$. To solve this, we again assign $x_1 = 1$ and solve for $x_2$.
\be
  \frac{k}{m}(1) - \frac{k}{m}x_2 \Longrightarrow x_2 = 1
\ee
So our eigenvector for $\lambda_2$ is $\begin{bmatrix} 1 \\ 1 \\ \end{bmatrix}$. The eigendecomposition equation for the mass-scaled Hessian becomes,
\be
  \bk' = \textbf{U} \Lambda \textbf{U}^{\dagger} =
  \begin{bmatrix}
    -1 & 1 \\
    1 & 1 \\
  \end{bmatrix}
  \begin{bmatrix}
    \frac{3k}{m} & 0 \\
    0 & \frac{k}{m} \\
  \end{bmatrix}
  \begin{bmatrix}
    -1 & 1 \\
    1 & 1 \\
  \end{bmatrix}^{\dagger} .
\ee
Remember that each column of \textbf{U} is the ith eigenvector that corresponds to the ith eigenvalue.

The normal mode coordinates and momenta are then defined from the mass-scaled coordinates and momenta.
\be \label{eq:ex_sol}
  \begin{split}
    \tilde{\br} &= \textbf{U}^{\dagger} \br' =
    \begin{bmatrix}
      -1 & 1 \\
      1 & 1 \\
    \end{bmatrix}^{\dagger}
    \begin{bmatrix}
      \sqrt{m} r_1 \\
      \sqrt{m} r_2 \\
    \end{bmatrix}
    =
    \begin{bmatrix}
       \sqrt{m} (r_2 - r_1) \\
       \sqrt{m} (r_1 + r_2) \\
    \end{bmatrix}
    =
    \begin{bmatrix}
      \tilde{r}_1 \\
      \tilde{r}_2 \\
    \end{bmatrix}
    = \sqrt{m}
    \begin{bmatrix}
        (r_2 - r_1) \\
        (r_1 + r_2) \\
    \end{bmatrix}
    \\
    \tilde{\bp} &= \textbf{U}^{\dagger} \bp'
    \begin{bmatrix}
      -1 & 1 \\
      1 & 1 \\
    \end{bmatrix}^{\dagger}
    \begin{bmatrix}
      \frac{1}{\sqrt{m}} p_1 \\
      \frac{1}{\sqrt{m}} p_2 \\
    \end{bmatrix}
    =
    \begin{bmatrix}
      \frac{1}{\sqrt{m}} (p_2 - p_1) \\
      \frac{1}{\sqrt{m}} (p_1 + p_2) \\
    \end{bmatrix}
    =
    \begin{bmatrix}
      \tilde{p}_1 \\
      \tilde{p}_2 \\
    \end{bmatrix}
    =
    \frac{1}{\sqrt{m}}
    \begin{bmatrix}
     (p_2 - p_1) \\
     (p_1 + p_2) \\
    \end{bmatrix}
  \end{split}
\ee
And with this, we have effectively turned our original coupled system into a system of independent harmonic oscillators!
In other words, in the normal mode basis, we can think of our system as having 2 normal modes: one where both masses oscillate with frequency $\omega_1 = \sqrt{\frac{3k}{m}}$ and another where both masses oscillate with frequency $\omega_2 = \sqrt{\frac{k}{m}}$.
Also, the position and momenta of both masses in each normal mode can be described with $(\tilde{r_1},\tilde{p_1})$ and $(\tilde{r_2},\tilde{p_2})$.

You may wonder why the position and momenta do not look sinusoidal. This is because we never explicitly defined $r_1$ and $r_2$. However, we expect both masses to osciallte sinusoidally so we can assume that the solutions will be sinusoidal. 

The analysis described in this lecture is simply transforming from one coordinate system to another so that whenever we are given $r_1, r_2, \dots$, for any system, we can immediately use equation (12) to find the coordinates in the normal mode basis. All we need to find is the matrix \textbf{U} that does this.
\end{document}
