\documentclass{article}
%=============================================================================80
%	                          Packages                                     %
%==============================================================================%
% Packages
\usepackage[utf8]{inputenc}
\usepackage{graphicx}
\usepackage{amsmath}
\usepackage{amssymb}
\usepackage{braket}
\usepackage{float}
\usepackage{subcaption}
\usepackage[margin=0.7in]{geometry}
\usepackage[version=4]{mhchem}
%==============================================================================%
%                           User-Defined Commands                              %
%==============================================================================%
% User-Defined Commands
\newcommand{\be}{\begin{equation}}
\newcommand{\ee}{\end{equation}}
\newcommand{\benum}{\begin{enumerate}}
\newcommand{\eenum}{\end{enumerate}}
\newcommand{\pd}{\partial}
\newcommand{\dg}{\dagger}
%==============================================================================%
%      Just for Lecture 15                                                     %
%==============================================================================%
\newcommand{\br}{\textbf{r}}
\newcommand{\bp}{\textbf{P}}
\newcommand{\bm}{\textbf{M}}
\newcommand{\bk}{\textbf{K}}
%==============================================================================%
%                             Title Information                                %
%==============================================================================%
\title{Chem237: Lecture 15}
\date{5/9/18}
\author{Alan Robledo}
%==============================================================================%
\begin{document}

\maketitle

\section{Classical Normal Mode Analysis}
%==============================================================================%
%    Somebody add an intro
%==============================================================================%

\subsection{Harmonic Approximation}
We start by writing the hamiltonian for our system of interest with N degrees of freedom,
\be
  H = \sum_{i = 1}^{N} \frac{p^2_i}{2 m} + V(r_1, \dots, r_N)
\ee
where our coorindate vector is $\br = \begin{bmatrix} r_1 \\ \vdots \\ r_N \end{bmatrix}$.
We can invoke the \textbf{Harmonic approximation}, which says that the potential $V(\br)$ can be expressed as a taylor expansion about a minimum $r_o$
\be
  V(r) = V(r_o) + \sum_j \Big( \frac{\partial V}{\partial r_j} \Big) (r - r_o) + \frac{1}{2} \sum_{i j} r_i K_{i j} r_j + \dots
\ee
where $\bk$ is called the \textbf{Hessian} and is defined as
\be
  k_{i j} = \frac{\partial^2 V}{\partial r_i r_j} .
\ee
Since our potential is expanded about a minimum, we can say that $\frac{\partial V(r_o)}{\partial r_i} = 0$.
If we neglect higher order terms and assume that $V(r_o) \approx 0$, then we obtain an approximate potential
\be
  V(r) \approx \frac{1}{2} \sum_{i j} r_i K_{i j} r_j .
\ee
The hamiltonian can then be written in terms of matrices as
\be
  \textbf{H} \approx \frac{1}{2} \bp^T \bm^{-1} \bp + \frac{1}{2} \textbf{r}^{T} \bk \textbf{r}
\ee
We have effectively turned our system into one where each degree of freedom can be approximated as a harmonic oscillator.
The equations of motion for each degree of freedom can be solved by using Hamilton's equations:
\be
  \begin{split}
    \dot{p}_i &= - \frac{\partial H}{\partial r_i} \\
    \dot{r}_i &= \frac{\partial H}{\partial p_i} \\
  \end{split}
\ee
However, this can be difficult when we have a system of coupled oscillators.

\subsection{Normal Mode Coordinates}
One way to get around this problem is by transforming our original coordinates into a set of normal mode coordinates.
This decouples all of our degrees of freedom and turns our system into one of N independent harmonic oscillators, each with a normal mode frequency $\omega_i$.
If our mass matrix \textbf{M} is not diagonal, we could perform an eigenvalue decomposition and use the corresponding eigenvectors to find a diagonal matrix for \bm, which would just be a matrix of its eigenvalues.
Instead, we will introduce mass scaled coordinates to get rid of our mass matrix from the hamiltonian.
These coordinates can be defined as:
\be
  \begin{split}
    r'_i = \sqrt{m_i} r_i \quad &\Longleftrightarrow \quad \br' = \bm^{1/2} \br \\
    p'_i = \frac{1}{\sqrt{m_i}} p_i \quad &\Longleftrightarrow \quad \bp' = \bm^{-1/2} \bp \\
    K'_{i j} = \frac{1}{\sqrt{m_i m_j}} K_{i j} \quad &\Longleftrightarrow \quad \bk' = \bm^{-1/2} \bk \bm^{1/2}
  \end{split}
\ee
So H becomes,
\be
  H = \frac{1}{2} \bk'^T \bk + \frac{1}{2} \br'^T \bk' \br'
\ee
where \bk' is now our mass-scaled Hessian.

\subsection{}
\end{document}
